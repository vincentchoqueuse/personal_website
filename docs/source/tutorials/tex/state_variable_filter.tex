\documentclass[tikz,14pt,border=10pt]{standalone}
\usepackage{textcomp}
\usetikzlibrary{shapes,arrows}
\begin{document}
% Definition of blocks:
\tikzset{%
  block/.style    = {draw, thick, rectangle, minimum height = 3em,
    minimum width = 3em},
  sum/.style      = {draw, circle, node distance = 2cm}, % Adder
  gain/.style      = {draw, isosceles triangle, node distance = 2cm}, % Adder
  nodecircle/.style    = {draw, circle, node distance = 2cm, fill=black}, % Adder
  input/.style    = {coordinate}, % Input
  output/.style   = {coordinate} % Output
}
% Defining string as labels of certain blocks.
\newcommand{\suma}{\Large$+$}
\newcommand{\inte}{$\displaystyle \int$}
\newcommand{\derv}{\huge$\frac{d}{dt}$}

\begin{tikzpicture}[auto, thick, node distance=2cm, >=triangle 45]
\draw
	% Drawing the blocks of first filter :
	node at (0,0)[right=-3mm]{\Large \textopenbullet}
    node at (-1,0) {$x[n]$}
	node [input, name=input1] {} 
	node [sum, right of=input1] (suma1) {\suma}
    node[left of=suma1, node distance=0.5cm, yshift=0.5cm] (){+}
    node[left of=suma1, node distance=0.5cm, yshift=-0.5cm] (){-}
    node [sum, right of=suma1] (suma2) {\suma}
    node[left of=suma2, node distance=0.5cm, yshift=0.5cm] (){+}
    node[left of=suma2, node distance=0.5cm, yshift=-0.5cm] (){-}
    node [nodecircle, right of=suma2, node distance=1cm] (node2) {}
    node[above of=node2](yh){$y_h[n]$}
    node [gain, right of=suma2] (gain1) {} 
    node[above of=gain1, node distance=0.5cm]{$\alpha_1$}
    node [sum, right of=gain1] (suma3) {\suma}
    node[left of=suma3, node distance=0.5cm, yshift=0.5cm] (){+}
    node[left of=suma3, node distance=0.5cm, yshift=-0.5cm] (){+}
    node [nodecircle, right of=suma3, node distance=4cm] (node0) {}
    node[above of=node0](yb){$y_b[n]$}
    node [gain, right of=suma3, node distance=6cm] (gain2) {} 
    node[above of=gain2, node distance=0.5cm]{$\alpha_1$}
    node [sum, right of=gain2] (suma4) {\suma}
    node[left of=suma4, node distance=0.5cm, yshift=0.5cm] (){+}
    node[left of=suma4, node distance=0.5cm, yshift=-0.5cm] (){+}
    node [nodecircle, right of=suma4, node distance=4cm] (node1) {} 
    node [right of=suma4, node distance=6cm] (out) {$y_l[n]$};

    % Joining blocks. 
    % Commands \draw with options like [->] must be written individually
	\draw[->](input1) -- node {}(suma1);
 	\draw[->](suma1) -- node {} (suma2);
	\draw[->](suma2) -- node {} (gain1);
    \draw[->](gain1) -- node {} (suma3);
	\draw[->](suma3) -- node {} (gain2);
    \draw[->](gain2) -- node {} (suma4);
    \draw[->](suma4) -- node {} (out);
	% Adder


\draw
	node[below of=suma3,xshift=2cm] [block] (z1){$z^{-1}$};
\draw[->](node0) |- node {} (z1);
\draw[->](z1) -| node {} (suma3);
\draw[->](node0) -- node {} (yb);
\draw[->](node2) -- node {} (yh);

\draw
	node[below of=z1] [block](z2){$z^{-1}$}
    node [gain, left of=z2, rotate=180] (gain3) {} node[above of=gain3, node distance=0.5cm]{$\alpha_2$};

\draw[->](node0) |- node {} (z2);
\draw[-](z2) -- node {} (gain3);
\draw[->](gain3) -| node {} (suma2);

\draw
	node[below of=z2] [block](z3){$z^{-1}$};


\draw
	node[below of=suma4,xshift=2cm] [block] (z4){$z^{-1}$};
    \draw[-](node1) |- node {} (z4);
    \draw[->](z4) -| node {} (suma4);
	
	\draw[-](node1) |- node {} (z3);
    \draw[->](z3) -| node {} (suma1);


\end{tikzpicture}
\end{document}



\end{document}